\documentclass{amsart}
\usepackage{fontspec}
\usepackage{xunicode}
\setmainfont[Mapping=tex-text,Ligatures=TeX]{CMU Serif}
\usepackage{polyglossia}
\setdefaultlanguage{russian}
\usepackage{amsthm,amsmath,amsfonts,amssymb}
\usepackage{enumerate}
\usepackage[colorlinks = true]{hyperref}
\usepackage{listings}
\usepackage[left=1cm,right=2cm,
    top=2cm,bottom=2cm,bindingoffset=0cm]{geometry}

\begin{document}

  \section{Грамматика языка L}

  Здесь описана грамматика языка L. Степень формальности и строгости немного опущена для простоты изложения.
  Определения идентификатров, литералов, бинарных операторов и т.п. взяты из домашнего задания про лексер.

  Основными понятиями в грамматике языка L являются объявление функции (function definition), вызов функции (function 
  call), инструкция (statement) и выражение (expression).

  Программа на языке L представляет собой набор из нескольких объявлений функций (возможно, ни одного) и инструкции.
    
  \subsection{Объявления функций}
  Объявление функции выглядит так:

  \begin{lstlisting}
  <function definition> : <name> '(' (<arg> (',' <arg>)*)? ')' '{' <statement> '}'
  \end{lstlisting}

  \subsection{Вызовы функций}
  Вызов функции выглядит так:

  \begin{lstlisting}
  <function call> : <name> '(' (<arg expr> (',' <arg expr>)*)? ')'
  \end{lstlisting}


  \subsection{Инструкции}
  Инструкция может быть одной из следующих:

  \begin{itemize}
      \item Присваивание
      \item Вызов функции
      \item Несколько инструкций, разделенных точкой с запятой
      \item Вызов функции
      \item Вывод в выходной поток
      \item Чтение из входного потока
      \item Цикл "пока"
      \item Условное ветвление
  \end{itemize}

  \begin{lstlisting}
  <statement> : <variable name> ':=' <expression> 
              | <function call>
              | <statement> (';' <statement>)+
              | 'write' <expression>
              | 'read' <variable name>
              | 'while' <expression> '{' <statement> '}'
              | 'if' <expression> '{' <statement> '}' ('else' '{' <statement> '}')?
  \end{lstlisting}

  \subsection{Выражения}
  Выражение может быть одним из следующих:

  \begin{itemize}
      \item Выражение, взятое в скобки
      \item Вызов функции
      \item Взятие значения переменной
      \item Целое число
      \item Число с плавающей точкой
      \item Применение бинарной операции к двум выражениям
  \end{itemize}

  \begin{lstlisting}
  <expression> : '(' <expression> ')'
               | <function call>
               | <variable name> 
               | <decimal integer literal>
               | <decimal floating point literal>
               | <expression> <binary operation> <expression>
  \end{lstlisting}

\end{document}
